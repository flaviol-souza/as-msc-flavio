\begin{frame}
	\frametitle{Sample frame title}
	This is a text in second frame. 
	For the sake of showing an example.
	
	\begin{itemize}
		\item<1-> Text visible on slide 1
		\item<2-> Text visible on slide 2
		\item<3> Text visible on slide 3
		\item<4-> Text visible on slide 4
	\end{itemize}
	
\end{frame}

\begin{slide}{Slide Title}
	You can see a list of items below. \pause \\
	There are commands to make them appear sequentially
	\begin{itemize}[type=1]
		\item<2> This is an item
		\item<3> Second item
		\item<4> Third item
	\end{itemize}
\end{slide}

\begin{frame}
	\frametitle{Sample frame title}
 
	In this slide, some important text will be \alert{highlighted} beause it's important. Please, don't abuse it.
 
	\begin{block}{Remark}
		Sample text
	\end{block}
	 
	\begin{alertblock}{Important theorem}
		Sample text in red box
	\end{alertblock}
 
	\begin{examples}
		Sample text in green box. "Examples" is fixed as block title.
	\end{examples}
\end{frame}

\begin{frame}
	\frametitle{Two-column slide}
	
	\begin{columns}
		
		\column{0.5\textwidth}
		This is a text in first column.
		$$E=mc^2$$
		\begin{itemize}
			\item First item
			\item Second item
		\end{itemize}
		
		\column{0.5\textwidth}
		This text will be in the second column
		and on a second tought this is a nice looking
		layout in some cases.
	\end{columns}
\end{frame}

\begin{frame}[fragile]\frametitle{Linguagem de programa\c c\~ao}
	Para mostrar c\'odigos de linguagem de programa\c c\~ao use o pacote \verb|minted|.
	
	Exemplo de c\'odigo Java.
	
	\begin{javacode}
		public class HelloWorldApp {
			public static void main (String args[])
			{
				System.out.println("Hello World!");
			}
		}
	\end{javacode}
	
	Para compilar com o pacote \verb|minted| \'e necess\'ario usar o comando \verb|-shell-escape| pelo terminal.
	
	\begin{minted}[bgcolor=lightgray!20]{bash}
		pdflatex -shell-escape minted01.tex
		ou
		latexmk -pdf -shell-escape minted01.tex
	\end{minted}
\end{frame}

\begin{frame}\frametitle{Inserindo figuras}
	Figuras devem ser inseridas no formato PNG, JPG ou PDF.
	
	\begin{figure}[h]
		\centering
		\includegraphics[height=0.6\paperheight]{figuras/figCoordEsf02}
		%\includegraphics[height=6cm]{figCoordEsf02}
		\caption{Sistema de coordenadas esf\'ericas.}\label{figCoordEsf02}
	\end{figure}
\end{frame}
Com o pacote TikZ podemos inserir figuras desenhadas em TikZ.
% Frame 10: figuras TikZ
\begin{frame}\frametitle{Figuras TikZ}
	Figuras feitas com TikZ.
	
	\begin{figure}[h]
		\centering
		\input{figuras/integral}
		\caption{Integral.}\label{figintegral}
	\end{figure}
\end{frame}

\begin{frame}\frametitle{Tabelas}
	\begin{table}
		\centering
		\begin{tabular}{cclrr}
			\toprule
			ID & Quant & Produto & Unit & Total\\
			\midrule
			1 & 2 & manga     & 3,00 & 6,00\\
			2 & 7 & laranja   & 1,20 & 8,40\\
			3 & 5 & banana    & 3,50 & 17,50\\
			4 & 3 & melancia  & 8,00 & 24,00\\
			5 & 4 & abacaxi   & 4,00 & 16,00\\
			\midrule
			Total &   &      &      & 71,90\\
			\bottomrule
		\end{tabular}
		\caption{Lista de compras}
	\end{table}
\end{frame}

\begin{frame}\frametitle{Transi\c c\~ao}
	Um pequeno exemplo de transi\c c\~ao.
	
	\pause
	
	\begin{enumerate}[a)]
		\item<2-< primeiro;
		\item<3-< segundo;
		\item<4-< terceiro.
	\end{enumerate}
\end{frame}