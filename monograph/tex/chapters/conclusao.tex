A computação em nuvem popularizou um serviço de comercialização se de capacidade computacional na qual a infraestrutura, plataforma ou \textit{software} são ofertados como produto sob demanda e onde os recursos são elásticos, despertando o interesse tanto da comunidade acadêmica quanto da indústria. Atualmente a maioria das grandes soluções de sistemas computacionais são compostas por \textit{mult-tiers}, inclusive quando se refere a aplicações web, devido à flexibilidade de escalabilidade. Para essas aplicações, o planejamento de capacidade é um requisito crítico para determinar a quantidade de recursos exigido para garantia de QoS. No entanto, o planejamento de capacidade é usualmente uma decisão de longo prazo em que, e os recursos são determinados por critérios estáticos. Desta forma, os recursos podem revelar uma sobrecarga em situações de perturbação, mesmo que os níveis de QoS esteja dentro da faixa aceitável para a carga estacionária. %Isso ocorre, devido a utilização da avaliação de desempenho, que é comumente destinada a responder a perguntas estáticas como qual o limite do sistema mediante a uma carga estacionaria imposta. 

Em sistemas que apresentam dinâmica acentuada, a avaliação de desempenho deve considerar que os períodos de regime transiente são importantes. Junto aos mecanismos de elasticidade dos recursos sob demanda, vem a necessidade do autogerenciamento dos recursos. Existem vários trabalhos disponíveis na literatura que lidam e tratam da gestão dos recursos computacionais. Neste trabalho há interesse na modelagem do sistema na forma de uma representação analítica capaz de reproduzir o comportamento dinâmico do sistema, onde o período transiente tem grande colaboração e impacto na política de gerenciamento dos recursos. Um diferencial em relação as abordagens convencionais o objetivo de determinar como a capacidade do sistema em lidar com a variação da carga de trabalho, ao invés de, o desempenho apenas com a carga de trabalho estacionaria.

Mediante a uma arquitetura conceitual proposta por \citeonline{Lourenco2015} e \citeonline{Edwin2015} propõe uma metodologia que descreve e especifica os passos para modelar um sistema computacional por meio de um \textit{benchmark}. No entanto, não foram encontrados \textit{benchmarks} que estimulem a dinâmica do sistema e que permitem uma avaliação em regime transiente, bem como não foram identificados \textit{benchmarks} que sigam a especificação de requisitos proposta por \citeonline{Lourenco2015}.
Este trabalho apresentou uma extensão de um \textit{benchmark}, o Bench4Q, capaz de modelar a sua carga de trabalho de tal maneira a estimular o sistema a apresentar sua dinâmica através da carga. Essa extensão segue um dos requisitos MESC, o modulo \textit{Demand}, proposto por \citeonline{Lourenco2015}, que se restringe a modulação da carga de trabalho, acrescendo-a de provisões para gerar perturbações programadas. Essa extensão, obedece ao padrão de implementação e usabilidade nativas do \textit{benchmark} Bench4Q. A extensão é provida de uma interface gráfica que possibilita a modelagem da carga através da inserção de parâmetros.

Para atingir o objetivo proposto, foi necessária a alteração da carga de trabalho nativa do Bench4Q. Essa modificação resultou na modulação da carga, possibilitando a geração de Degrau Positivo com a carga. Este modelo de carga tem como característica a alteração da sua potência de maneira brusca e repentina. Também é possível gerar um Degrau Negativo que tem efeito oposto ao Degrau Positivo, ou seja, o modelo da carga tem por característica a queda repentina de sua potência. Outra modelagem de carga que a extensão permite é a geração de uma Onda Quadrada, onde existe uma alternância entre os dois modelos descritos anteriormente. 
Por se tratar de um trabalho de extensão de \textit{benchmark} difundido e grande complexidade, as alterações efetuadas trouxeram dificuldades relacionadas a implementação. A falta de uma documentação técnica gerou grande esforço no entendimento e compreensão da implementação original, necessitando muitas vezes a depuração do código para o claro entendimento do seu fluxo de funcionamento tão quando os módulos envolvidos.

O projeto apresentou exemplos práticos das modelagem das cargas propostas (Degrau Positivo, Degrau Negativo e Onda Quadrada) através dos resultados gerados pelo próprio \textit{benchmark}. Os resultados da presente pesquisa foram adequado, na medida em que responderam positivamente ao objetivo do trabalho. O trabalho contemplou uma bateria de experimentos práticos em um ambiente controlado \textit{mult-tier}. Na execução das fases de experimentos, foi elaborado o planejamento do experimento, a coleta de dados e juntamente a análises dos resultados obtidos, resultando em um conjunto de contribuições para a área de pesquisa:
\begin{itemize}
	\item A elaboração de uma documentação padronizada da mesma forma que a original do Bench4Q, contando com versão em inglês que se encontra no apêndice \ref{chapter:documentacao};
	
	\item Impacto da carga modulada: mediante a modulação da carga através da extensão, é possível excitar o sistema a apresentar a sua dinâmica, contribuindo para trabalhos que tem por necessidade a modelagem do sistema e do seu comportamento dinâmico;
	
	\item Dinâmica entre camadas: ao se tratar de um sistema de multicamadas, foi possível perceber, juntamente com a modulação da carga, a dinâmica intrínseca entre as camadas do sistema. Os efeitos combinados de atrasos intrínsecos, ainda que pequenos, e sua propagação por todo as camadas interligados geraram um comportamento dinâmico significante e apreciável;
	
	\item Métrica que mascara: por consequência da dinâmica inerente ao sistema multicamadas, foi possível vislumbrar que nem toda métrica apresenta a realidade quando se lida com um sistema multicamadas. Neste trabalho, foi possível observar esse fato na métrica tempo de resposta.
\end{itemize} 

O presente trabalho foi desenvolvido em consequência aos interesse de estudo da pesquisa de \citeonline{Edwin2015}, e em paralelo a outra iniciativa de \citeonline{Lourenco2015} que especificam a identificação de capacidade dinâmica em sistemas computacionais e como tratá-las com técnicas de modelagem de sistemas. Os resultados do presente trabalho contribuem para ambos os projetos com a disponibilização de um \textit{benchmark} que auxilia na modelagem de sistemas computacionais dinâmicos.

O presente trabalho implementa um dos requisitos do modelo conceitual de referência MEDC. A extensão permite que um \textit{benchmark}, focado na avaliação de desempenho estacionária, seja capaz de uma avaliação não-estacionaria sendo está a principal contribuição do trabalho.

\section{Trabalhos Futuros}
O presente trabalho de mestrado contribuiu para o desenvolvimento de técnicas e estudos interessados na dinâmica do sistema. Todavia, existe uma gama de trilhas a serem exploradas até que a importância da dinâmica de sistemas computacionais tenha maior apreço, como nos casos das ciências e engenharias. Consequentemente este trabalho não finaliza as possibilidades de estudo relacionadas e outros estudos podem ser desenvolvidos a partir dos resultados e constatações identificadas dentre os são:
\begin{itemize}
	\item \textbf{Novas formas de pertubação:} com base nas proposta de \cite{Hellerstein2004} existem outras funções que auxiliam a excitar o sistema a apresentarem a sua dinâmica. 
	
	\item \textbf{Avaliação da dinâmica em sistemas multicamadas:} com o presente trabalho, foi possível observar a dinâmica entre as camadas do sistema, entretanto o presente trabalho não cobre com um planejamento de experimentos utilizando métodos estatísticos por meio de avaliações de desempenho
	
	\item \textbf{Avaliação de métricas de planejamento de recursos em ambiente multicamadas:} este trabalho também revelou que existem métricas que podem mascarar a realidade, neste contexto é importante uma análise das principais métricas de diferentes técnicas para o planejamento de recursos sob um ambiente de multicamadas
\end{itemize} 	
